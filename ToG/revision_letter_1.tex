%%%%%%%%%%%%%%%%%%%%%%%%%%%%%%%%%%%%%%%%%
% Thin Formal Letter
% LaTeX Template
% Version 2.0 (7/2/17)
%
% This template has been downloaded from:
% http://www.LaTeXTemplates.com
%
% Author:
% Vel (vel@LaTeXTemplates.com)
%
% Originally based on an example on WikiBooks 
% (http://en.wikibooks.org/wiki/LaTeX/Letters) but rewritten as of v2.0
%
% License:
% CC BY-NC-SA 3.0 (http://creativecommons.org/licenses/by-nc-sa/3.0/)
%
%%%%%%%%%%%%%%%%%%%%%%%%%%%%%%%%%%%%%%%%%

%----------------------------------------------------------------------------------------
%	DOCUMENT CONFIGURATIONS
%----------------------------------------------------------------------------------------

\documentclass[10pt]{letter} % 10pt font size default, 11pt and 12pt are also possible

\usepackage{geometry} % Required for adjusting page dimensions

%\longindentation=0pt % Un-commenting this line will push the closing "Sincerely," to the left of the page

\geometry{
	paper=a4paper, % Change to letterpaper for US letter
	top=3cm, % Top margin
	bottom=1.5cm, % Bottom margin
	left=4.5cm, % Left margin
	right=4.5cm, % Right margin
	%showframe, % Uncomment to show how the type block is set on the page
}

\usepackage[T1]{fontenc} % Output font encoding for international characters
\usepackage[utf8]{inputenc} % Required for inputting international characters

\usepackage{stix} % Use the Stix font by default

\usepackage{microtype} % Improve justification
\usepackage{hyperref}
\hypersetup{
	colorlinks=true,
	linkcolor=blue,
	filecolor=magenta,      
	urlcolor=cyan,
}


%----------------------------------------------------------------------------------------
%	YOUR NAME & ADDRESS SECTION
%----------------------------------------------------------------------------------------

\signature{Mohammed SALEM} % Your name for the signature at the bottom

\address{University of Mascara \\ Algeria\\ salem@univ-mascara.dz}

%----------------------------------------------------------------------------------------

\begin{document}

%----------------------------------------------------------------------------------------
%	ADDRESSEE SECTION
%----------------------------------------------------------------------------------------

\begin{letter}{Professor Julian Togelius \\ Editor in Chief, IEEE Transactions on Games} % Name/title of the addressee

%----------------------------------------------------------------------------------------
%	LETTER CONTENT SECTION
%----------------------------------------------------------------------------------------
\opening{Subject: Revision and resubmission of manuscript TCIAIG-2019-0090\\
	\\	
\textbf{Dear Editor of  IEEE Transactions on Games journal,}}


We would like to thank you for the opportunity to resubmit a revised copy of our paper entitled ``Overtaking Uncertainty with Evolutionary TORCS controllers: Combining adaptive BLX-$\alpha$ Operator and Grand Prix Selection''. 

We would also like to take this opportunity to express our gratitude to the reviewers for their positive feedback and helpful comments for correction or modification. Thus, the manuscript has been revised addressing the reviewers' comments and suggestions.

We believe this has resulted in an improved revised manuscript, which you will find uploaded alongside this document. In it, changes and new contents have been marked in red color.

In addition we have appended the reviewers' comments below in this letter and responded to them point by point, indicating exactly how we have addressed each concern or request, and describing the changes we have made. The revisions have been approved by all the authors and I personally have again been chosen as the corresponding author. 
We sincerely hope the revised manuscript worth being accepted for publication in the Journal.

Thank you for your time and consideration.

I look forward to here your reply.

\vspace{2\parskip} % Extra whitespace for aesthetics
\closing{Sincerely yours,}
\vspace{2\parskip} % Extra whitespace for aesthetics

\ps{P.S. You can find additional information attached to this letter.} % Postscript text, comment this line to remove it







\newpage

Reviewer Comments, Author Responses and Manuscript Changes
\begin{enumerate}

% #############################################################################

\item {\bf \underline{ Reviewer 1 Comments}}\\
	\begin{itemize}
		\item {\bf Comment 1: Fitnessless:
		\begin{quote}
			I would be especially interested in a general definition of the
			proposed selection method, so it could be applied in other
			contexts. Furthermore, it is claimed that the Grand Prix Selection is
			a fitnessless method. I am not quite fond of this term since the
			ranking after driving a race can be considered as fitness as well,
			e.g. doesn't be successful in playing the game itself define a fitness
			function? Even if it is not arbritarily chosen, the environment's
			designer or the person who defines the task inherently defined a goal
			as well. In the introduction, it is even mentioned that the system
			uses a "kind of fitness selection". To better distinguish fitnessless
			methods from traditional ones, I would be useful to provide a
			definition.
		\end{quote}
	}
		\item {\bf Response:} 
		Both you and the other reviewers commented on the fitnessless. 
		We have added an explanation of why we call it fitnessless, since it
		does not give an invariant and individual score (or fitness) to every
		individual, it's simply a way of keeping track the positions in the
		podium, and, in that sense, it's simply a way of tallying the result
		of a (repeated) tournament selection; the individuals don't have a
		fitness {\em before} entering Grand Prix Selection and we give the
		result of the tournaments based on that fitness. The selection is what
		yields a score that does not really have existence outside the
		selection procedure.
		
		We have added this text to the paper in order to clarify what we
		understand by fitnessless:
		{\em Although this selection uses a score that could be assimilated to a fitness, it is actually an extension of a tournament selection policy since it creates tournaments of several individuals, and ``scores'' them according to how they fare in these races. This is not actually a fitness, since it is not intrinsic to the individual. It is indeed equivalent to, in a $n$-tournament selection that is repeated several times, giving a score of $n$ to the first, $n-1$ to the second, and then using this for selection. That score is, thus, not a fitness but actually a way of keeping track of the position of the individual in the different tournaments it has participated; since, in this context, we have no way of evaluating (i.e. assigning a fitness) to a controller but only a way to compare them, we call this approach {\em fitnessless}, as it was called, for instance, in [18].}
% Antonio - I tried to add a bibliographic item and it failed, so maybe this would be easier as in any other response letter.

% ------------------------------------------------------------------------

		\item {\bf Comment 2: Runtime comparison:
		\begin{quote}
			In the evaluation, the evolutionary approaches are compared based on 		
			the number of generations and the resulting fitness. Since it is also 
			mentioned that the Grand Prix Selection is a time-consuming selection 
			method (in terms of processing time) it would be interesting to do 	
			similar performance comparisons based on runtime of the algorithm.
		\end{quote}
		}
		\item {\bf Response:} Following the reviewer suggestion, we have added a new comparison at the end of the experiments section, including a new table showing the running time of every approach, as well as an indication about the generation in which the best solution was found in every case.

% ------------------------------------------------------------------------

		\item {\bf Comment 3: Other rivals from the competition:
		\begin{quote}
			Furthermore, the evaluation consists of multiple variants of the
			proposed algorithm and version from the author's previous papers. It 
			also includes the baseline bots of the TORCS framework and a single bot 
			from the competition. Could you mention why the bot PSRI was chosen and 
			why the evaluation is not using any of the other bots of the 
			competition? To have a broad comparison it would be useful to include 
			other bots as well or argue why the comparison may not be useful 
			according to the focus of this paper.		
		\end{quote}
		}
		\item {\bf Response:} The reviewer is right, indeed we tried to contact to many other authors of bots which participated in the competition in order to include them in the comparison. Unfortunately, we just received a positive answer from the creators of PSRI bot.

% ------------------------------------------------------------------------

		\item {\bf Comment 4: Typos and Grammar errors:
		\begin{quote}
			- Abstract: experimens -> experiments
			- how good a bot or controller is by playing the game -> how good a bot
	  		  or controller is, is by playing the game
			- $\alpha$ parameter allows to control the search... -> The $\alpha$ 
           parameter allows to control the search
		\end{quote}
		}
		\item {\bf Response:} We thank the reviewer for mentioning these errors which we fixed now. 

% ------------------------------------------------------------------------

		\item {\bf Comment 5: Presentation Style about the linebreak of the word CG-Track 1 and sorting the Tables values.}
		\item {\bf Response:} Thank you so much for catching these glaring and confusing presentation style errors. We have corrected them in previous and new tables in the manuscript.

% ------------------------------------------------------------------------

\item {\bf Comment 6: Presentation of graph skewness and kurtosis
\begin{quote}
			Figure 4: instead of plotting certain points in time consider plotting
			the distribution of skewness and its variance over time and do the same 
			with kurtosis. Consider skewness to be a probability distribution that 
			changes over time and plot its mean and its variance per timestep.
			From the picture I assume, that the mean value converges to 0 (or a 
			small positive number) and the variances reduce over time.
		\end{quote}
		}
              \item {\bf Response:} We have revised the text related to this comment and added a conclusion:
                \begin{quote}
                  ... this was an objective of this GPS procedure, and we consider it has been achieved
                \end{quote}

 The reviewer comment really nails the conclusion; however, our
 intention was not so much to check the evolution, specially of the
 variance, but to see what was the final result compared to the
 initial one (and intermediate), that is, how efficiently skewness and
 kurtosis is reduced, thus reducing uncertainty and also improving
 results. Using fitness, it was relatively easy to select an
 individual just by chance, and that is shown by the high kurtosis and
 skewness even in the last generations. By using a fitnessless
 selection procedure based on races, only individuals who solidly
 perform better (small 1st, 2nd and 3rd degree moments of their
 fitness score) will be able to succeed; by representing mainly the
 values 2nd and 3rd degree moments in the latest generations, we prove
 what we were looking for. Representing the evolution for every
 generation would mean a lot of additional computation would have to
 be made (because every point involves a different number of races),
 and the conclusion is not very likely to vary. We might obtain some
 insight on when exactly that variation occur, either in early stages
 or final stages of evolution, but while that might have some
 intrinsic value, it's rather beside the point for this paper. We hank
 anyway the reviewer for this comment.
   \end{itemize}

\newpage

% ##############################################################################

\item {\bf \underline{ Reviewer 2 Comments}}\\
	\begin{itemize}
		\item {\bf  Comment 1:  Revise writing and text style:\\
		\begin{quote}
			Abstract:
			"...Grand Prix Selection (GPS), which should be able..." -> I recommend 
			being assertive here, if it does reduce uncertainty demonstrated in 
			your results.

			"...experimens..." -> "experiments"
			
			Page 1:
			"Driving a car is a problem..." -> in a video game or in real life? 
			Both?
			"...that cars do not crash, and are able..." -> no comma
			"...so that the car wins in as many..." -> "...the car wins as many..."
			"...with increasing success." -> needs citation
			"...as well as a not so good exploration..." -> "suboptimal"

			Page 2:
			"...This operator works, at the same time,..." -> "This operator works 
			at the same time"
			"..and let the designer.." -> "lets"
			"Thus, in this paper we are testing..." -> this paragraph is one 		
			sentence long, I think it would benefit from being split up to be more 
			clear what you are introducing.
			\end{quote}}
			\item {\bf Response:} All these suggested corrections have been done in 	
 			the manuscript. Moreover, the whole text has been revised 
			again carefully. Changes have been marked in red color in order to ease 
			finding them in the new version of the manuscript. 

% ------------------------------------------------------------------------

		\item {\bf  Comment 2:  How the fitnessless method or GPS is first introduced in the introduction:\\
			                \begin{quote}
				According to the authors, simulating solo-races is
				not a reliable way of evaluating a car
				controller. Instead, they suggest to use a relative
				performance measure based on ranks in 10-car races,
				thus also being able to capture the controller
				performance in the presence of other cars. I would
				argue this is not actually a different selection
				operator, it is just a different fitness function. It
				was therefore confusing to me why the authors chose
				to call their approach "fitnessless" - they are still
				using a score that is compared. I would suggest to
				either clarify this further or pick different
				terminology.
			\end{quote}}
			\item {\bf Response:}
		We thank the reviewer for this suggestion. In fact, it can be
		considered an alternative fitness function; however, usually methods that don't use a fitness score that is independent of the selection procedure are considered fitnessless; we also call it {\em fitnessless} as opposed to the fitness function we give to the same individual in solo races, when that's the method used in some generations in the paper. Just to clarify, this {\em score} is simply a writedown of the position of the car in the different races, in the
		same way that if we were running repeated tournament selections we would tally up the number of wins or loses. It's also called fitnessless because there's no absolute number that can be called that way, no crisp number that tells you if a car controller is better than other. We can only compare them (and only partially), and see how they
		run in different races; this {\em fitness} (which should be called more properly a score) is the result of the selection process, not the other way round, the selection process is the result of the fitness of the individuals participating in it. This is similar to a coevolutionary strategy, and has been called traditionally {\em fitnessless} (for instance in [18]), which is what we do here.
		As the reviewer suggests, we have added a clarification (and a
		reference) in the following paragraph:\\
		{\em Thus, in this paper we are testing the best approaches we have found all together in an algorithm, considering a kind of fitnessless selection, which we have called \textit{Grand Prix Selection} (GPS). Although this selection uses a score that could be assimilated to a fitness, it is actually an extension of a tournament selection policy since it creates tournaments of several individuals, and ``scores'' them according to how they fare in these races. This is not actually a fitness, since it is not intrinsic to the individual. It is indeed equivalent to, in a $n$-tournament selection that is repeated several times, giving a score of $n$ to the first, $n-1$ to the second, and then using this for selection. That score is, thus, not a fitness but actually a way of keeping track of the position of the individual in the different tournaments it has participated; since, in this context, we have no way of evaluating (i.e. assigning a fitness) to a controller but only a way to compare them, we call this approach {\em fitnessless}, as it was called, for instance, in [18].}

% ------------------------------------------------------------------------

		\item {\bf  Comment 3: My understanding of Figure 4 is that all the points come from a single experimental evolutionary run. Because GPS creates pockets of "relative fitness" due to the nature of how it evaluates controllers against one another, I think that this paper would benefit from showing the results of multiple runs, to ensure that GPS consistently shows improvement over the GFC method.}
                \item {\bf Response:} A reference to the specific behavior in the shown run has been eliminated. On the other hand, the behavior is the same in different runs; since representing them would distract the reader from the main point, which is that fitness-less methods reduce skewness and kurtosis in a more consistent way than those that use fitness functions, we have simply added this sentence:
                  \begin{quote}
                    Although we have represented a single run here, results for other runs are similar.
                    \end{quote}

		
    \end{itemize}

\newpage

% ##############################################################################\\

\item {\bf \underline{ Reviewer 3 Comments}}\\
	\begin{itemize}
	\item {\bf  Comment 1: It was therefore confusing to me why the authors chose to call their approach "fitnessless" - they are still using a score that is compared. I would suggest to either clarify this further or pick different terminology.}
	\item {\bf Response:} 
	Thank you for this comment so we have added an explanation of why we call it fitnessless, since it does not give an invariant and individual score (or fitness) to every
individual, it's simply a way of keeping track the positions in the
podium, and, in that sense, it's simply a way of tallying the result
of a (repeated) tournament selection; the individuals don't have a
fitness {\em before} entering Grand Prix Selection and we give the
result of the tournaments based on that fitness. The selection is what
yields a score that does not really have existence outside the
selection procedure.

We have added this text to the paper in order to clarify what we
understand by fitnessless:
{\em Although this selection uses a score that could be assimilated to a fitness, it is actually an extension of a tournament selection policy since it creates tournaments of several individuals, and ``scores'' them according to how they fare in these races. This is not actually a fitness, since it is not intrinsic to the individual. It is indeed equivalent to, in a $n$-tournament selection that is repeated several times, giving a score of $n$ to the first, $n-1$ to the second, and then using this for selection. That score is, thus, not a fitness but actually a way of keeping track of the position of the individual in the different tournaments it has participated; since, in this context, we have no way of evaluating (i.e. assigning a fitness) to a controller but only a way to compare them, we call this approach {\em fitnessless}, as it was called, for instance, in [18].}\\
% ------------------------------------------------------------------------

		\item {\bf  Comment 2:	There is just one track that is used for training, and the same track is used for evaluation as well (in addition to one other track)}:
		\item {\bf Response:}
		We thank the reviewer for this important remark about the number of used tracks in the training process. In the initial manuscript we used the Alpine 2 track due to how onerous is the proposed fitnessless optimization. 
		Now, the authors have considered this reviewer's comment and added another track, `Wheel 2', to train the controllers; and `Street 1' track for testing purposes.\\
		We have also multiplied the number of races, running 10 instead of 5 for each track.

		Thus, we have carried out new experiments considering two training tracks: the previously used `Alpine 2' track and a newly introduced called `Wheel 2' track, which is a Suzuka F1 inspired track with more challenging difficulties, hairpin turns and long straight segments. It is longer than Alpine 2 (See Figure 3 in the paper).\\
		Furthermore, another track has been added for testing the controller, named `Street 1' track. This controller has many long speed segments, difficult sharp turns. In addition, the track is two meters larger than the others  three considered tracks which increase the overtaking possibilities.\\
		These changes imply updating all the Tables and Figures of the Experiments and Results section. All the modifications have been marked in red color in the text.\\
		
% ------------------------------------------------------------------------
		 
		\item {\bf   Comment 3:	About the analysis of the noise section} However, it could have been clearer how many individuals were picked and how often they were evaluated.
		\item {\bf Response:}
		Authors selects randomly $20$ individuals from the $60$ that compose the population.
		We have  added this detail in red in the paper:
		{\em This is why we have computed skewness and kurtosis for a sample of 20 from the 60 individuals of the population}\\	

% ---------------------------------------------------------------------
		\item {\bf   Comment 4: There were also some statements I was missing some form of evidence for:}
			\begin{itemize}
			\item	"successful racing will only need these sensor values" (2)
					\item {\bf Response:} 
					Designing a successful racing fuzzy controller   could be done using only the three distances  between the car and the track limits: FRONT, MAX5 and MAX10	since they give enough information to avoid collisions, drive as close as possible to the
					center of the track, and drive as fast as possible when needed.\\
					
					
			\item	the approach could generalise to other types of controllers (6)
					\item {\bf Response:} 
					We wrote down in the paper that {\em 'The result could be generalized to any kind
					of controller, as long as the evolution process has the possibility to explore the
					space of controllers in the same, efficient, way'}
				Indeed, that main result of the paper is that we can tune the parameters of  TORCS fuzzy controller with an evolutionary process  using a sort of a fitnessless selection. This finding could be generalized to any other TORCS controller when an evolution parameters tuning is needed.\\
				
			\item	reduced diversity is good (1) -> usually, the opposite is true in my experience
					\item {\bf Response:} *** TO ANSWER ***
			\end{itemize}

% ------------------------------------------------------------------------

		\item {\bf 	  Comment 5: While the paper is generally well-structured, there are several problems with the writing, sometimes making the message unclear. Examples are:}
			\begin{itemize}
			\item "A fitness is substituted by a podium"
					\item {\bf Response:} 
					That means that instead of using a certain fitness value for an individual , we replace it by a ranking after several races with other controllers.\\
					
			\item "use neuroevolution instead of evolving" -> both are evolutionary approaches?
					\item {\bf Response:} *** TO ANSWER ***\\
			\item What does "map the trajectory" mean?
					\item {\bf Response:} *** TO ANSWER ***\\
			\item It is not clear whether the statistics about the articles cited are about TORCS or not? And if they aren't, why are they relevant?
					\item {\bf Response:} *** TO ANSWER ***\\
					
			\item "the higher diversification factor which it means"
					\item {\bf Response:} : It was a typographic error, we deleted the words 'which it means' 
			\end{itemize}
	
%\\
10
32 33 19 
%24
 45 ------------------------------------------------------------------------

	\item {\bf   Comment 6: Minor language comments: :} 
				\begin{itemize}
				\item 	"driving a car is a problem" -> It is not clear at this point that you are talking about an optimisation problem
				\item {\bf Response:}  
				Thank you for the remark, the sentence has been changed in the introduction in red color:
				{\em Driving a simulated car is a problem in which ... }
				\\
				\item 	"eventually some of them are decided" -> Eventually makes it sound like there is a sequence of events here?
				\item {\bf Response:} *** TO CHECK ***\\
				Here , 'eventually' in the sense of possibly, or probably

				\item 	"in this environment is quite hard" -> very vague
				\item {\bf Response:} *** TO ANSWER ***\\

				\item 	"on the other hand" should only be used in conjunction with "on the one hand"
				\item {\bf Response:} *** TO CHECK **\\
			Yes, the reviewer is right but we can also use 'on the other hand' to introduce the second of {\bf two } contrasting points, facts, or ways of looking at something without 'on the one hand'.\\
			\url{https://dictionary.cambridge.org/grammar/british-grammar/contrasts}	\\
%%%%%%%%  or this link  
%%https://www.englishgrammar.org/hand/#:~:text=On%20the%20other%20hand%20is%20not%20a%20conjunction.,commas%20to%20set%20it%20off.				%%%% I don't know if we can include links about grammar rules
				\\
				\item 	"previsely"?
				\item {\bf Response:} It was a typographic error which has been fixed. The correct word is `precisely'.
			
			\end{itemize}
	
\end{itemize}
\end{enumerate}
 

%%%%%%%%%%%%%%%%%%%%%%%%%%%%%%%%%%%%%




%----------------------------------------------------------------------------------------

\end{letter}

\end{document}

%%%%%%%%%%%%%%%%%%%%%%%%%%%%%%%%%%%%%%%%%
% Thin Formal Letter
% LaTeX Template
% Version 2.0 (7/2/17)
%
% This template has been downloaded from:
% http://www.LaTeXTemplates.com
%
% Author:
% Vel (vel@LaTeXTemplates.com)
%
% Originally based on an example on WikiBooks 
% (http://en.wikibooks.org/wiki/LaTeX/Letters) but rewritten as of v2.0
%
% License:
% CC BY-NC-SA 3.0 (http://creativecommons.org/licenses/by-nc-sa/3.0/)
%
%%%%%%%%%%%%%%%%%%%%%%%%%%%%%%%%%%%%%%%%%

%----------------------------------------------------------------------------------------
%	DOCUMENT CONFIGURATIONS
%----------------------------------------------------------------------------------------

\documentclass[10pt]{letter} % 10pt font size default, 11pt and 12pt are also possible

\usepackage{geometry} % Required for adjusting page dimensions

%\longindentation=0pt % Un-commenting this line will push the closing "Sincerely," to the left of the page

\geometry{
	paper=a4paper, % Change to letterpaper for US letter
	top=3cm, % Top margin
	bottom=1.5cm, % Bottom margin
	left=4.5cm, % Left margin
	right=4.5cm, % Right margin
	%showframe, % Uncomment to show how the type block is set on the page
}

\usepackage[T1]{fontenc} % Output font encoding for international characters
\usepackage[utf8]{inputenc} % Required for inputting international characters

\usepackage{stix} % Use the Stix font by default

\usepackage{microtype} % Improve justification

%----------------------------------------------------------------------------------------
%	YOUR NAME & ADDRESS SECTION
%----------------------------------------------------------------------------------------

\signature{Mohammed SALEM} % Your name for the signature at the bottom

\address{University of Mascara \\ Algeria\\ salem@univ-mascara.dz}

%----------------------------------------------------------------------------------------

\begin{document}

%----------------------------------------------------------------------------------------
%	ADDRESSEE SECTION
%----------------------------------------------------------------------------------------

\begin{letter}{Professor Julian Togelius \\ Editor in Chief, IEEE Transactions on Games} % Name/title of the addressee

%----------------------------------------------------------------------------------------
%	LETTER CONTENT SECTION
%----------------------------------------------------------------------------------------
\opening{Subject: Revision and resubmission of manuscript TCIAIG-2019-0090\\
	\\	
\textbf{Dear Editor,  IEEE Transactions on Games journal ,}}


We would like to thank you for the opportunity to resubmit a revised copy of our paper entitled “Overtaking Uncertainty with Evolutionary TORCS controllers: Combining adaptive BLX-$\alpha$ Operator and Grand Prix Selection”. We would also like to take this opportunity to express our thanks to the reviewers for the positive feedback and helpful comments for correction or modification.
We believe have resulted in an improved revised manuscript, which you will find uploaded alongside this document. The manuscript has been revised to address the reviewer comments where the changes are mentioned in red color.\\
I have appended the reviewer comments alongside this letter and responded to them individually, indicating exactly how we addressed each concern or problem and describing the changes we have made. The revisions have been approved by all the authors and I have again been chosen as the corresponding author. The changes are marked in red in the paper as you requested, and the revised manuscript is attached to this email message

We very much hope the revised manuscript is accepted for publication in Journal.

Thank you for your time and consideration.

I look forward to your reply.

\vspace{2\parskip} % Extra whitespace for aesthetics
\closing{Sincerely yours,}
\vspace{2\parskip} % Extra whitespace for aesthetics

\ps{P.S. You can find additional information attached to this letter.} % Postscript text, comment this line to remove it







\newpage

Reviewer Comments, Author Responses and Manuscript Changes
\begin{enumerate}
\item {\bf \underline{ Reviewer 1 Comments}}\\
	\begin{itemize}
		\item {\bf Comment 1: Fitnessless:
		\begin{quote}
			I would be especially interested in a general definition of the
			proposed selection method, so it could be applied in other
			contexts. Furthermore, it is claimed that the Grand Prix Selection is
			a fitnessless method. I am not quite fond of this term since the
			ranking after driving a race can be considered as fitness as well,
			e.g. doesn't be successful in playing the game itself define a fitness
			function? Even if it is not arbritarily chosen, the environment's
			designer or the person who defines the task inherently defined a goal
			as well. In the introduction, it is even mentioned that the system
			uses a "kind of fitness selection". To better distinguish fitnessless
			methods from traditional ones, I would be useful to provide a
			definition.
		\end{quote}
	}
		\item {\bf Response:} 
		Both you and the other reviewers commented on the fitnessless. 
		We have added an explanation of why we call it fitnessless, since it
		does not give an invariant and individual score (or fitness) to every
		individual, it's simply a way of keeping track the positions in the
		podium, and, in that sense, it's simply a way of tallying the result
		of a (repeated) tournament selection; the individuals don't have a
		fitness {\em before} entering Grand Prix Selection and we give the
		result of the tournaments based on that fitness. The selection is what
		yields a score that does not really have existence outside the
		selection procedure.
		
		We have added this text to the paper in order to clarify what we
		understand by fitnessless:
		{\em Although this selection uses a score that could be assimilated to a fitness, it's actually an extension of a tournament selection policy since it creates tournaments of several individuals, and ``scores'' them according to how they fare in these races. This is not actually a fitness, since it's not intrinsic to the individual. It's equivalent to, in a $n$-tournament selection that is repeated several times, giving a score of $n$ to the first, $n-1$ to the second, and then using this for selection. That score is, thus, not a fitness but actually a way of keeping track of the position of the individual in the different tournaments it's participated.}
		\item {\bf Comment 2:Typos and Grammar errors:}
		\item {\bf Response:} We thank the reviewer for mentioning these errors which we fixed now. 
		\item {\bf Comment 3:Presentation Style about the linebreak of the word CG-Track 1 and sorting the Tables values.}
		\item {\bf Response:} Thank you so much for catching these glaring and confusing presentation style errors, which we have now corrected.
              \end{itemize}




\item {\bf \underline{ Reviewer 2 Comments}}\\
	\begin{itemize}
		\item {\bf  Comment 1:  How the fitnessless method or GPS is first introduced in the introduction:\\
			                \begin{quote}
				According to the authors, simulating solo-races is
				not a reliable way of evaluating a car
				controller. Instead, they suggest to use a relative
				performance measure based on ranks in 10-car races,
				thus also being able to capture the controller
				performance in the presence of other cars. I would
				argue this is not actually a different selection
				operator, it is just a different fitness function. It
				was therefore confusing to me why the authors chose
				to call their approach "fitnessless" - they are still
				using a score that is compared. I would suggest to
				either clarify this further or pick different
				terminology.
			\end{quote}}
			\item {\bf Response:}
		We thank the reviewer for this suggestion. In fact, it can be
		considered an alternative fitness function; however, usually methods that don't use a fitness score that is independent of the selection procedure are considered fitnessless; we also call it {\em fitnessless} as opposed to the fitness function we give to the same individual in solo races, when that's the method used in some generations in the paper. Just to clarify, this {\em score} is simply a writedown of the position of the car in the different races, in the
		same way that if we were running repeated tournament selections we would tally up the number of wins or loses. It's also called fitnessless because there's no absolute number that can be called that way, no crisp number that tells you if a car controller is better than other. We can only compare them (and only partially), and see how they
		run in different races; this {\em fitness} (which should be called 	more properly a score) is the result of the selection process, not the other way round, the selection process is the result of the fitness of the individuals participating in it. This is similar to a coevolutionary strategy, and has been called traditionally {\em fitnessless}, which is what we do here.
		As the reviewer suggests, we have added a clarification (and a
		reference) in the following paragraph:\\
		{\em Thus, in this paper we are testing the best approaches we have found all together in an algorithm, considering a kind of fitnessless selection, which we have called \textit{Grand Prix Selection} (GPS). Although this selection uses a score that could be assimilated to a fitness, it's actually an extension of a 	tournament selection policy since it creates tournaments of several individuals, and ``scores'' them according to how they fare in these races. This is not actually a fitness, since it's not intrinsic to
		the individual. It's equivalent to, in a $n$-tournament selection that is repeated several times, giving a score of $n$ to the first, $n-1$ to the second, and then using this for selection. That score is, thus, not a fitness but actually a way of keeping track of the position of the individual in the different tournaments it's participated; since, in this context, we have no way of evaluating (=assigning a fitness) to a controller but only a way to compare	them, we call this approach {\em fitnessless}, as it was called, for instance, in {\sc jaskowski2008winning}.}
		
		\item {\bf  Comment 2: Language errors :} 
			\item {\bf Response:} Revisited
	                \end{itemize}
\item {\bf \underline{ Reviewer 3 Comments}}\\
	\begin{itemize}
	\item {\bf  Comment 1: It was therefore confusing to me why the authors chose to call their approach "fitnessless" - they are still using a score that is compared. I would suggest to either clarify this further or pick different terminology.}\\
	\item {\bf Response:}
..
		\item {\bf  Comment 2:	There is just one track that is used for training, and the same track is used for evaluation as well (in addition to one other track)}:\\
		\item {\bf Response:}
				
		Authors have considered two other tracks: Wheel 2 which is a Suzuka F1 inspired track with more challenging difficulties and Street 1 track for testing purpose.\\
		They are multiplied the races number running the  10 races for each track.

		\item {\bf   Comment 3:	About the analysis of the noise section} However, it could have been clearer how many individuals were picked and how often they were evaluated.\\
		\item {\bf Response:}
		Authors selects randomly $20\%$ of the population size (60 individuals)
	
		\item {\bf   Comment 4: There were also some statements I was missing some form of evidence for:}
			\begin{itemize}
			\item	"successful racing will only need these sensor values" (2)\\
					\item {\bf Response:} ............................
			\item	the approach could generalise to other types of controllers (6)\\
					\item {\bf Response:} ..........................
			\item	reduced diversity is good (1) -> usually, the opposite is true in my experience\\
					\item {\bf Response:} ............................
			\end{itemize}

		\item {\bf 	  Comment 5: While the paper is generally well-structured, there are several problems with the writing, sometimes making the message unclear. Examples are:}
			\begin{itemize}
			\item "A fitness is substituted by a podium"\\
					\item {\bf Response:} ............................
			\item "use neuroevolution instead of evolving" -> both are evolutionary approaches?\\
					\item {\bf Response:} ............................
			\item What does "map the trajectory" mean?\\
					\item {\bf Response:} ............................
			\item It is not clear whether the statistics about the articles cited are about TORCS or not? And if they aren't, why are they relevant?\\
					\item {\bf Response:} ............................
			\item "the higher diversification factor which it means"\\
					\item {\bf Response:} ............................
			\end{itemize}
	

	\item {\bf   Comment 6: Minor language comments: :} 
				\begin{itemize}
				\item 	"driving a car is a problem" -> It is not clear at this point that you are talking about an optimisation problem\\
				\item {\bf Response:}  .......................
				\item 	"eventually some of them are decided" -> Eventually makes it sound like there is a sequence of events here?\\
				\item {\bf Response:} .......................
	

				\item 	"in this environment is quite hard" -> very vague\\
				\item {\bf Response:} ........................

				\item 	"on the other hand" should only be used in conjunction with "on the one hand"\\
				\item {\bf Response:} ........................
				\item 	"previsely"?\\
				\item {\bf Response:} ........................
			
			\end{itemize}
	
\end{itemize}
\end{enumerate}
 

%%%%%%%%%%%%%%%%%%%%%%%%%%%%%%%%%%%%%




%----------------------------------------------------------------------------------------

\end{letter}
 
\end{document}
